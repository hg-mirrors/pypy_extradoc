\documentclass[utf8x]{beamer}

% This file is a solution template for:

% - Talk at a conference/colloquium.
% - Talk length is about 20min.
% - Style is ornate.

\mode<presentation>
{
  \usetheme{Warsaw}
  % or ...

  %\setbeamercovered{transparent}
  % or whatever (possibly just delete it)
}


\usepackage[english]{babel}
\usepackage{listings}
\usepackage{ulem}
\usepackage{color}
\usepackage{alltt}

\usepackage[utf8x]{inputenc}


\newcommand\redsout[1]{{\color{red}\sout{\hbox{\color{black}{#1}}}}}

% or whatever

% Or whatever. Note that the encoding and the font should match. If T1
% does not look nice, try deleting the line with the fontenc.


\title{Loop-Aware Optimizations in PyPy’s Tracing JIT}

\author[Ardö, Bolz, Fijałkowski]{Håkan Ardö$^1$ \and \emph{Carl Friedrich Bolz}$^2$ \and Maciej Fijałkowski}
% - Give the names in the same order as the appear in the paper.
% - Use the \inst{?} command only if the authors have different
%   affiliation.

\institute[Lund, Düsseldorf]{
$^1$Centre for Mathematical Sciences, Lund University \and
$^2$Heinrich-Heine-Universität Düsseldorf, STUPS Group, Germany
}

\date{2012 DLS, 22nd of October, 2012}
% - Either use conference name or its abbreviation.
% - Not really informative to the audience, more for people (including
%   yourself) who are reading the slides online


% If you have a file called "university-logo-filename.xxx", where xxx
% is a graphic format that can be processed by latex or pdflatex,
% resp., then you can add a logo as follows:




% Delete this, if you do not want the table of contents to pop up at
% the beginning of each subsection:
%\AtBeginSubsection[]
%{
%  \begin{frame}<beamer>
%    \frametitle{Outline}
%    \tableofcontents[currentsection,currentsubsection]
%  \end{frame}
%}


% If you wish to uncover everything in a step-wise fashion, uncomment
% the following command: 

%\beamerdefaultoverlayspecification{<+->}


\begin{document}

\begin{frame}
  \titlepage
\end{frame}

\begin{frame}
  \frametitle{Why do tracing JITs work?}
  \begin{itemize}
      \item They are good at selecting interesting and common code paths
      \item both through user program and through the runtime
      \item the latter is particularly important for dynamic languages
          \pause
      \item traces are trivial to optimize
  \end{itemize}
\end{frame}

\begin{frame}
  \frametitle{Optimizing traces}
  \begin{itemize}
      \item Traces trivial to optimize, because there's no control flow
      \item most optimizations are one forward pass
      \item optimizers are often like symbolic executors
      \item can do optimizations that are untractable with full control flow
      \item XXX example
  \end{itemize}
\end{frame}

\begin{frame}
  \frametitle{Problems with this approach}
  \begin{itemize}
      \item most traces actually are loops
      \item naive foward passes ignore this bit of control flow optimization available
      \item how to fix that without sacrifing simplicity?
  \end{itemize}
\end{frame}

\begin{frame}
  \frametitle{Idea for solution}
  \begin{itemize}
      \item idea first proposed and implemented in LuaJIT by Mike Pall
      \item this talk presents the implementation of the same approach in RPython's tracing JIT
  \end{itemize}
  \pause
  \begin{block}{Approach}
      \begin{itemize}
          \item do a pre-processing step on the traces
          \item apply the unchanged forward-pass optimizations
          \item do some post-processing
          \item pre-processing is done in such a way that the normal optimizations become loop-aware
          \item intuition: give the optimizations a second iteration of context to work with
      \end{itemize}
  \end{block}
\end{frame}

\begin{frame}
  \frametitle{Pre-processing the loops}
  \begin{itemize}
      \item pre-processing does loop unrolling
      \item peels off one iteration of the loop, duplicating the trace
      \item the optimizations optimize both iterations together
      \item this yields loop-invariant code motion and related optimizations
  \end{itemize}
\end{frame}


\end{document}
