\section{Related Work}

Promotion is a concept that we have already explored in other contexts. Psyco is
a run-time specialiser for Python that uses promotion (called ``unlift'' in
\cite{DBLP:conf/pepm/Rigo04}). However, Psyco is a manually written JIT, is
not applicable to other languages and cannot be retargetted.

Moreover, the idea of promotion of is a generalization of \emph{Polymorphic
  Inline Caches} \cite{hoelzle_optimizing_1991}, as well as the idea of using
runtime feedback to produce more efficient code
\cite{hoelzle_type_feedback_1994}.

PyPy-style JIT compilers are hard to write manually, thus we chose to write a
JIT generator.  Tracing JIT compilers \cite{gal_hotpathvm_2006} also gives
good results but are much easier to write, making the need for an automatic
generator less urgent.  However so far tracing JITs have less general
allocation removal techniques, which makes them get less speedup in a dynamic
language with boxing.  Another difference is that tracing JITs concentrate on
loops, which makes them produce a lot less code.  This issue will be addressed
by future research in PyPy.

The code generated by tracing JITs code typically contains guards; in recent research
\cite{gal_incremental_2006} on Java, these guards' behaviour is extended to be
similar to our promotion.  This has been used twice to implement a dynamic
language (JavaScript), by Tamarin\footnote{{\tt
http://www.mozilla.org/projects/tamarin/}} and in \cite{chang_efficient_2007}.

There has been an enormous amount of work on partial evaluation for compiler
generation. A good introduction is given in \cite{Jones:peval}. However, most of
it is for generating ahead-of-time compilers, which cannot produce very good
performance results for dynamic languages.

However, there is also some research on runtime partial evaluation. One of the
earliest examples is Tempo for C
\cite{DBLP:conf/popl/ConselN96,DBLP:conf/dagstuhl/ConselHNNV96}. However, it is
essentially an offline specializer ``packaged as a library''; decisions about
what can be specialized and how are pre-determined.

Another work in this direction is DyC \cite{grant_dyc_2000}, another runtime
specializer for C.  Specialization decisions are also pre-determined, but
``polyvariant program-point specialization'' gives a coarse-grained equivalent
of our promotion.  Targeting the C language makes higher-level specialization
difficult, though (e.g.\ \texttt{mallocs} are not removed).

Greg Sullivan introduced "Dynamic Partial Evaluation", which is a special
form of partial evaluation at runtime \cite{sullivan_dynamic_2001} and describes
an implementation for a small dynamic language based on lambda calculus. This
work is conceptually very close to our own.
% XXX there are no performance figures, we have no clue how much of this is
% implemented. not sure how to write this

Our algorithm to avoid allocation of unneeded intermediate objects fits into
the research area of escape analysis: in comparison to advanced techniques
\cite{Blanchet99escapeanalysis}, \cite{Choi99escapeanalysis} our algorithm is
totally simple-minded, but it is still useful in practise.

\commentout{
\section{Future work}

XXX to be written
}

\section{Conclusion}

high level structure:

1. We presented PyPy JIT generator 

2. interpreters can exploit the JIT generator by placing few hints here and
there to guide the generation of code

3. generated JITs produce fast code thanks to \textbf{promotion} (our first
contribution) and automatic unboxing

4. we presented a way to implement promotion/flexswitches on top of .NET

5. our second contribution is to show that the idea of JIT layering can work
well, as for some cases we can get performances even better than C\#
