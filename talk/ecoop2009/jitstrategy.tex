\section{Effective JIT compilation of dynamic languages}

Writing efficient compilers for dynamic languages is hard.  Since these
languages are dynamically typed, usually the compiler does not have enough
information to produce efficient code, but instead it has to insert a lot of
runtime checks to select the appropriate implementation for each operation.

By generating code at runtime, JIT compilers can exploit some extra knowledge
compared to traditional static compilers.  However, we need to take special
care to choose a strategy for JIT compilation that lets the compiler to take
the best of this advantage.

Most JIT compilers for dynamic languages around (such as IronPython, Jython,
JRuby \anto{XXX insert some reference}) compile code at the method
granularity.  If on one hand they can exploit some of the knowledge gathered
at runtime (e.g. the types of method parameters), on the other hand they can
do little to optimize most of the operations inside, because their behaviour
depends on informations that are not available at compile time, because
e.g. the global state of the program can change at runtime. \anto{e.g., we can
  add/remove methods to classes, etc. Should we insert some example here?}

JIT compilers generated by PyPy solve this problem by delaying the compilation
until they know all the informations needed to generate efficient code.  If at
some point the JIT compiler does not know about something it needs, it
generates a callback into itself.  

Later, when the generated code is executed, the callback is hit and the JIT
compiler is restarted again.  At this point, the JIT knows exactly the state
of the program and can exploit all this extra knowledge to generate highly
efficient code.  Finally, the old code is patched and linked to the newly
generated code, so that the next time the JIT compiler will not be invoked
again.  As a result, \textbf{runtime and JIT-compile time are continuously
  intermixed}. \anto{maybe we should put the translation/compile/run time
  distinction here}

XXX: insert an example of flexswitch
