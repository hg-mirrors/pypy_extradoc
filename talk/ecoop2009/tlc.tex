\section{The TLC language}

In this section, we will briefly describe \emph{TLC}, a simple dynamic
language that we developed to exercise our JIT compiler generator.  As most of
dynamic languages around, \emph{TLC} is implemented through a virtual machine
that interprets a custom bytecode. Since our main interest is in the runtime
performance of the VM, we did not implement the parser nor the bytecode
compiler, but only the VM itself.

TLC provides four different types:
\begin{enumerate}
\item Integers
\item \lstinline{nil}, whose only value is the null value
\item Objects
\item Lisp-like lists
\end{enumerate}

Objects represent a collection of named attributes (much like JavaScript or
Self) and named methods.  At creation time, it is necessary to specify the set
of attributes of the object, as well as its methods.  Once the object has been
created, it is not possible to add/remove attributes and methods.

The virtual machine is stack-based, and provides several operations:

\begin{itemize}
\item \textbf{Stack manipulation}: standard operations to manipulate the
  stack, such as \lstinline{PUSH}, \lstinline{POP}, \lstinline{SWAP}, etc.
\item \textbf{Flow control} to do conditional and unconditional jumps.
\item \textbf{Arithmetic}: numerical operations on integers, like
  \lstinline{ADD}, \lstinline{SUB}, etc.
\item \textbf{Comparisons} like \lstinline{EQ}, \lstinline{LT},
  \lstinline{GT}, etc.
\item \textbf{Object-oriented}: operations on objects: \lstinline{NEW},
  \lstinline{GETATTR}, \lstinline{SETATTR}, \lstinline{SEND}.
\item \textbf{List operations}: \lstinline{CONS}, \lstinline{CAR},
  \lstinline{CDR}.
\end{itemize}

Obviously, not all the operations are applicable to all objects. For example,
it is not possible to \lstinline{ADD} an integer and an object, or reading an
attribute from an object which does not provide it.  Being a dynamic language,
the VM needs to do all these checks at runtime; in case one of the check
fails, the execution is simply aborted.

\anto{should we try to invent a syntax for TLC and provide some examples?}
\cfbolz{we should provide an example with the assembler syntax}

