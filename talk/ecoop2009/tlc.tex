\section{The TLC language}

In this section, we will briefly describe \emph{TLC}, a simple dynamic
language that we developed to exercise our JIT compiler generator and that will
be used as a running example in this paper. The design goal of the language is
to be very simple (the interpreter of the full language consists of about 600
lines of RPython code) but to still have the typical properties of dynamic
languages that make them hard to compile. \emph{TLC} is implemented with a small
interpreter that interprets a custom bytecode instruction set. Since our main
interest is in the runtime performance of the interpreter, we did not implement
the parser nor the bytecode compiler, but only the interpreter itself.

TLC provides four different types:
\begin{enumerate}
\item Integers
\item \lstinline{nil}, whose only value is the null value
\item Objects
\item Lisp-like lists
\end{enumerate}

Objects represent a collection of named attributes (much like JavaScript or
Self) and named methods.  At creation time, it is necessary to specify the set
of attributes of the object, as well as its methods.  Once the object has been
created, it is possible to call methods and read or write attributes, but not
to add or remove them.

The interpreter for the language is stack-based and uses bytecode to represent
the program. It provides the following bytecode instructions:

\begin{itemize}
\item \textbf{Stack manipulation}: standard operations to manipulate the
  stack, such as \lstinline{POP}, \lstinline{PUSHARG}, \lstinline{SWAP}, etc.
\item \textbf{Flow control} to do conditional (\lstinline{BR_COND}) and
  unconditional (\lstinline{BR}) jumps.
\item \textbf{Arithmetic}: numerical operations on integers, like
  \lstinline{ADD}, \lstinline{SUB}, etc.
\item \textbf{Comparisons} like \lstinline{EQ}, \lstinline{LT},
  \lstinline{GT}, etc.
\item \textbf{Object-oriented} operations: \lstinline{NEW},
  \lstinline{GETATTR}, \lstinline{SETATTR}, \lstinline{SEND}.
\item \textbf{List operations}: \lstinline{CONS}, \lstinline{CAR},
  \lstinline{CDR}.
\end{itemize}

Obviously, not all the operations are applicable to all types. For example,
it is not possible to \lstinline{ADD} an integer and an object, or reading an
attribute from an object which does not provide it.  Being dynamically typed,
the interpreter needs to do all these checks at runtime; in case one of the check
fails, the execution is simply aborted.

\subsection{TLC properties}
\label{sec:tlc-properties}

Despite being very simple and minimalistic, \lstinline{TLC} is a good
candidate as a language to test our JIT generator, as it has some of the
properties that makes most of current dynamic languages so slow:

\begin{itemize}

\item \textbf{Stack based interpreter}: this kind of interpreter requires all the operands to be
  on top of the evaluation stack.  As a consequence programs spend a lot of
  time pushing and popping values to/from the stack, or doing other stack
  related operations.  However, thanks to its simplicity this is still the
  most common and preferred way to implement interpreters.

\item \textbf{Boxed integers}: integer objects are internally represented as
  an instance of the \lstinline{IntObj} class, whose field \lstinline{value}
  contains the real value.  By having boxed integers, common arithmetic
  operations are made very slow, because each time we want to load/store their
  value we need to go through an extra level of indirection.  Moreover, in
  case of a complex expression, it is necessary to create many temporary
  objects to hold intermediate results.

\item \textbf{Dynamic lookup}: attributes and methods are looked up at
  runtime, because there is no way to know in advance if and where an object
  have that particular attribute or method.
\end{itemize}


\subsection{TLC example}

As we said above, TLC exists only at bytecode level; to ease the development
of TLC programs, we wrote an assembler that generates TLC bytecode. Figure \ref{fig:tlc-abs}
shows a simple program that computes the absolute value of
the given integer.  In the subsequent sections, we will examine step-by-step 
how the generated JIT compiler manages to produce a fully optimized version of it.

\begin{figure}[h]
\begin{center}
\begin{lstlisting}
main:             # stack: []
    PUSHARG       #        [n]
    PUSH 0        #        [n, 0]
    LT            #        [0 or 1]
    BR_COND neg

pos:              #        []
    PUSHARG       #        [n]
    RETURN

neg:
    PUSH 0        #        [0]
    PUSHARG       #        [0,n]
    SUB           #        [-n]
    RETURN
\end{lstlisting}
\caption{The TLC bytecode for computing the absolute value of a function}
\label{fig:tlc-abs}
\end{center}
\end{figure}

\commentout{

\anto{I think this example could cause confusion, because the synatx is TOO
  similar to (R)Python, hence in the next section it could be unclear which is
  the leves of the examples}

Since reading TLC programs at bytecode level is hard, in this paper we will
use an invented Python-like syntax to describe examples, although we need to
remind that the actual programs are written in the assembler language showed
above. Thus, the example above can be written in the following way:

\begin{lstlisting}
def main(n):
    if n<0:
        return -n
    return n
\end{lstlisting}
}
