\documentclass{sigplanconf}

\usepackage{ifthen}
\usepackage{fancyvrb}
\usepackage{color}
\usepackage{wrapfig}
\usepackage{ulem}
\usepackage{xspace}
\usepackage{relsize}
\usepackage{epsfig}
\usepackage{amssymb}
\usepackage{amsmath}
\usepackage{amsfonts}
\usepackage[utf8]{inputenc}
\usepackage{setspace}

\usepackage{listings}

\usepackage[T1]{fontenc}
\usepackage[scaled=0.81]{beramono}


\definecolor{commentgray}{rgb}{0.3,0.3,0.3}

\lstset{
  basicstyle=\ttfamily\footnotesize,
  language=Python,
  keywordstyle=\bfseries,
  stringstyle=\color{blue},
  commentstyle=\color{commentgray}\textit,
  fancyvrb=true,
  showstringspaces=false,
  %keywords={def,while,if,elif,return,class,get,set,new,guard_class}
  numberstyle = \tiny,
  numbersep = -20pt,
}

\newboolean{showcomments}
\setboolean{showcomments}{false}
\ifthenelse{\boolean{showcomments}}
  {\newcommand{\nb}[2]{
    \fbox{\bfseries\sffamily\scriptsize#1}
    {\sf\small$\blacktriangleright$\textit{#2}$\blacktriangleleft$}
   }
   \newcommand{\version}{\emph{\scriptsize$-$Id: main.tex 19055 2008-06-05 11:20:31Z cfbolz $-$}}
  }
  {\newcommand{\nb}[2]{}
   \newcommand{\version}{}
  }

\newcommand\cfbolz[1]{\nb{CFB}{#1}}
\newcommand\toon[1]{\nb{TOON}{#1}}
\newcommand\anto[1]{\nb{ANTO}{#1}}
\newcommand\arigo[1]{\nb{AR}{#1}}
\newcommand\fijal[1]{\nb{FIJAL}{#1}}
\newcommand\pedronis[1]{\nb{PEDRONIS}{#1}}
\newcommand{\commentout}[1]{}

\newcommand{\noop}{}


\newcommand\ie{i.e.,\xspace}
\newcommand\eg{e.g.,\xspace}

\normalem

\let\oldcite=\cite

\renewcommand\cite[1]{\ifthenelse{\equal{#1}{XXX}}{[citation~needed]}{\oldcite{#1}}}

\definecolor{gray}{rgb}{0.5,0.5,0.5}

\begin{document}

\title{Efficiently Handling Guards in the low level design of RPython's tracing JIT}

\authorinfo{Carl Friedrich Bolz$^a$ \and David Schneider$^{a}$}
           {$^a$Heinrich-Heine-Universität Düsseldorf, STUPS Group, Germany
           }
           {XXX emails}

\conferenceinfo{VMIL'11}{}
\CopyrightYear{2012}
\crdata{}

\maketitle

\category{D.3.4}{Programming Languages}{Processors}[code generation,
incremental compilers, interpreters, run-time environments]

\terms
Languages, Performance, Experimentation

\keywords{XXX}

\begin{abstract}

\end{abstract}


%___________________________________________________________________________
\section{Introduction}


The contributions of this paper are:
\begin{itemize}
 \item
\end{itemize}

The paper is structured as follows:

\section{Background}
\label{sec:Background}

\subsection{RPython and the PyPy Project}
\label{sub:pypy}


The RPython language and the PyPy Project were started in 2002 with the goal of
creating a python interpreter written in a High level language, allowing easy
language experimentation and extension. PyPy is now a fully compatible
alternative implementation of the Python language, xxx mention speed. The
Implementation takes advantage of the language features provided by RPython
such as the provided tracing just-in-time compiler described below.

RPython, the language and the toolset originally developed to implement the
Python interpreter have developed into a general environment for experimenting
and developing fast and maintainable dynamic language implementations. xxx Mention
the different language impls.

RPython is built of two components, the language and the translation toolchain
used to transform RPython programs to executable units.  The RPython language
is a statically typed object oriented high level language. The language provides
several features such as automatic memory management (aka. Garbage Collection)
and just-in-time compilation. When writing an interpreter using RPython the
programmer only has to write the interpreter for the language she is
implementing.  The second RPython component, the translation toolchain, is used
to transform the program to a low level representations suited to be compiled
and run on one of the different supported target platforms/architectures such
as C, .NET and Java. During the transformation process
different low level aspects suited for the target environment are automatically
added to program such as (if needed) a garbage collector and with some hints
provided by the author a just-in-time compiler.



\subsection{PyPy's Meta-Tracing JIT Compilers}
\label{sub:tracing}

 * Tracing JITs
 * JIT Compiler
   * describe the tracing jit stuff in pypy
   * reference tracing the meta level paper for a high level description of what the JIT does
   * JIT Architecture
   * Explain the aspects of tracing and optimization

%___________________________________________________________________________


\section{Resume Data}
\label{sec:Resume Data}

* High level handling of resumedata
   * trade-off fast tracing v/s memory usage
   * creation in the frontendÂ
   * optimization
   * compression
   * interaction with optimization
   * tracing and attaching bridges and throwing away resume data
   * compiling bridges

% section Resume Data (end)

\section{Guards in the Backend}
\label{sec:Guards in the Backend}

* Low level handling of guards
   * Fast guard checks v/s memory usage
   * memory efficient encoding of low level resume data
   * fast checks for guard conditions
   * slow bail out

% section Guards in the Backend (end)

%___________________________________________________________________________


\section{Evaluation}
\label{sec:evaluation}

* Evaluation
   * Measure guard memory consumption and machine code size
   * Extrapolate memory consumption for guard other guard encodings
      * compare to naive variant
   * Measure how many guards survive optimization
   * Measure the of guards and how many of these ever fail

\section{Related Work}


\section{Conclusion}


\section*{Acknowledgements}

\bibliographystyle{abbrv}
\bibliography{paper}

\end{document}
