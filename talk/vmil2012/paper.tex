\documentclass{sigplanconf}

\usepackage{ifthen}
\usepackage{fancyvrb}
\usepackage{color}
\usepackage{wrapfig}
\usepackage{ulem}
\usepackage{xspace}
\usepackage{relsize}
\usepackage{epsfig}
\usepackage{amssymb}
\usepackage{amsmath}
\usepackage{amsfonts}
\usepackage[utf8]{inputenc}
\usepackage{setspace}

\usepackage{listings}

\usepackage[T1]{fontenc}
\usepackage[scaled=0.81]{beramono}


\definecolor{commentgray}{rgb}{0.3,0.3,0.3}

\lstset{
  basicstyle=\ttfamily\footnotesize,
  language=Python,
  keywordstyle=\bfseries,
  stringstyle=\color{blue},
  commentstyle=\color{commentgray}\textit,
  fancyvrb=true,
  showstringspaces=false,
  %keywords={def,while,if,elif,return,class,get,set,new,guard_class}
  numberstyle = \tiny,
  numbersep = -20pt,
}

\newboolean{showcomments}
\setboolean{showcomments}{false}
\ifthenelse{\boolean{showcomments}}
  {\newcommand{\nb}[2]{
    \fbox{\bfseries\sffamily\scriptsize#1}
    {\sf\small$\blacktriangleright$\textit{#2}$\blacktriangleleft$}
   }
   \newcommand{\version}{\emph{\scriptsize$-$Id: main.tex 19055 2008-06-05 11:20:31Z cfbolz $-$}}
  }
  {\newcommand{\nb}[2]{}
   \newcommand{\version}{}
  }

\newcommand\cfbolz[1]{\nb{CFB}{#1}}
\newcommand\toon[1]{\nb{TOON}{#1}}
\newcommand\anto[1]{\nb{ANTO}{#1}}
\newcommand\arigo[1]{\nb{AR}{#1}}
\newcommand\fijal[1]{\nb{FIJAL}{#1}}
\newcommand\pedronis[1]{\nb{PEDRONIS}{#1}}
\newcommand{\commentout}[1]{}

\newcommand{\noop}{}


\newcommand\ie{i.e.,\xspace}
\newcommand\eg{e.g.,\xspace}

\normalem

\let\oldcite=\cite

\renewcommand\cite[1]{\ifthenelse{\equal{#1}{XXX}}{[citation~needed]}{\oldcite{#1}}}

\definecolor{gray}{rgb}{0.5,0.5,0.5}

\begin{document}

\title{Efficiently Handling Guards in the low level design of RPython's tracing JIT}

\authorinfo{Carl Friedrich Bolz$^a$ \and David Schneider$^{a}$}
           {$^a$Heinrich-Heine-Universität Düsseldorf, STUPS Group, Germany
           }
           {XXX emails}

\conferenceinfo{VMIL'11}{}
\CopyrightYear{2012}
\crdata{}

\maketitle

\category{D.3.4}{Programming Languages}{Processors}[code generation,
incremental compilers, interpreters, run-time environments]

\terms
Languages, Performance, Experimentation

\keywords{XXX}

\begin{abstract}

\end{abstract}


%___________________________________________________________________________
\section{Introduction}


The contributions of this paper are:
\begin{itemize}
 \item 
\end{itemize}

The paper is structured as follows: 

\section{Background}
\label{sec:Background}

\subsection{RPython and the PyPy Project}
\label{sub:pypy}


\subsection{PyPy's Meta-Tracing JIT Compilers}
\label{sub:tracing}

 * Tracing JITs
 * JIT Compiler
   * describe the tracing jit stuff in pypy
   * reference tracing the meta level paper for a high level description of what the JIT does
   * JIT Architecture
   * Explain the aspects of tracing and optimization

%___________________________________________________________________________


\section{Resume Data}
\label{sec:Resume Data}

* High level handling of resumedata
   * trade-off fast tracing v/s memory usage
   * creation in the frontend 
   * optimization
   * compression
   * interaction with optimization
   * tracing and attaching bridges and throwing away resume data
   * compiling bridges

% section Resume Data (end)

\section{Guards in the Backend}
\label{sec:Guards in the Backend}

* Low level handling of guards
   * Fast guard checks v/s memory usage
   * memory efficient encoding of low level resume data
   * fast checks for guard conditions
   * slow bail out

% section Guards in the Backend (end)

%___________________________________________________________________________


\section{Evaluation}
\label{sec:evaluation}

* Evaluation
   * Measure guard memory consumption and machine code size
   * Extrapolate memory consumption for guard other guard encodings
      * compare to naive variant
   * Measure how many guards survive optimization
   * Measure the of guards and how many of these ever fail

\section{Related Work}


\section{Conclusion}


\section*{Acknowledgements}

\bibliographystyle{abbrv}
\bibliography{paper}

\end{document}
